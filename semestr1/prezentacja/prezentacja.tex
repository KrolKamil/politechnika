\documentclass[10pt]{beamer}

\usepackage{hyperref}

\usepackage{caption}

\usepackage{polski}
\usepackage[utf8]{inputenc}

\usetheme{metropolis}
\usepackage{appendixnumberbeamer}

\usepackage{booktabs}
\usepackage[scale=2]{ccicons}

\usepackage{pgfplots}
\usepgfplotslibrary{dateplot}

\usepackage{xspace}
\newcommand{\themename}{\textbf{\textsc{metropolis}}\xspace}

\title{Podstawy Kryptologii}
\subtitle{}
\date{\today}
\author{Kamil Król}
\institute{Politechnika Śląska, Wydział Matematyki Stosowanej}
% \titlegraphic{\hfill\includegraphics[height=1.5cm]{logo.pdf}}

\begin{document}

\maketitle

\begin{frame}{Spis treści}
  \setbeamertemplate{section in toc}[sections numbered]
  \tableofcontents[hideallsubsections]
\end{frame}

\section{Wprowadzenie}

\begin{frame}[fragile]{Kryptologia}
 
Kryptologia – dziedzina wiedzy o przekazywaniu informacji w sposób zabezpieczony przed niepowołanym dostępem.\\
Współcześnie kryptologia jest uznawana za gałąź zarówno matematyki, jak i informatyki.\\

Kryptologię dzieli się na:\\

\begin{itemize}
    \item kryptografię - czyli gałąź wiedzy o utajnianiu wiadomości;
    
    \item kryptoanalizę - czyli gałąź wiedzy o przełamywaniu zabezpieczeń
  \end{itemize}
  
\end{frame}
\begin{frame}[fragile]{Terminologia}
  Istotnym elementem technik kryptograficznych jest proces zamiany tekstu jawnego w szyfrogram (inaczej kryptogram); proces ten nazywany jest \textbf{szyfrowaniem}, a proces odwrotny, czyli zamiany tekstu zaszyfrowanego na powrót w możliwy do odczytania, \textbf{deszyfrowaniem}.
  
  
  Przez \textbf{szyfr} rozumiana jest para algorytmów służących do przeprowadzenia obu procesów. Wraz z algorytmami dodatkowo używa się \textbf{kluczy}, czyli pewnych niezbędnych parametrów, od których zależy wynik obu procesów. Innymi słowy: znajomość algorytmu i szyfrogramu bez dostępu do klucza nie pozwoli na odtworzenie tekstu jawnego.
\end{frame}
\begin{frame}[fragile]{Terminologia}
  Kryptografia nie zajmuje się jednak wyłącznie szyfrowaniem i deszyfrowaniem tekstów. Po pierwsze, dane przekazywane są najczęściej w postaci binarnej, co umożliwia również obróbkę takich danych jak dźwięk czy obraz; po drugie, równie ważne jak zapewnianie poufności danych jest ich \textbf{integralność} (niezmienność danych w czasie procesu), \textbf{uwierzytelnianie} (pewność co do ich pochodzenia) oraz \textbf{niezaprzeczalność} (nadawca nie może wyprzeć się faktu, że był nadawcą wiadomości).
  
  
  Ważnym terminem używanym w kryptografii jest \textbf{kryptosystem}, a więc system obejmujący zastosowane w danym wypadku szyfry, metody generowania kluczy, urządzenia wraz z oprogramowaniem oraz procedury ich użycia. Istotnym aspektem kryptosystemu jest jego bezpieczeństwo – odporność na ataki kryptologiczne.
\end{frame}

\section{Szyfr Cezara}

\begin{frame}{Opis szyfru}
	\textbf{Szyfr Cezara} (zwany też szyfrem przesuwającym, kodem Cezara lub przesunięciem Cezariańskim) – jedna z najprostszych technik szyfrowania.
	
	 Jest to rodzaj szyfru podstawieniowego, w którym każda litera tekstu jawnego (niezaszyfrowanego) zastępowana jest inną, oddaloną od niej o stałą liczbę pozycji w alfabecie, literą (szyfr monoalfabetyczny), przy czym kierunek zamiany musi być zachowany.
	 
	  Nie rozróżnia się przy tym liter dużych i małych. Nazwa szyfru pochodzi od Juliusza Cezara, który prawdopodobnie używał tej techniki do komunikacji ze swymi przyjaciółmi.
\end{frame}


\begin{frame}{Przykład}
	Sposób szyfrowania może być przedstawiony za pomocą diagramu dwóch ciągów z odpowiadającymi sobie kolejnymi literami alfabetu. Te same litery drugiego ciągu są przesunięte względem ciągu pierwszego o określoną liczbę pozycji, zwaną parametrem przesunięcia (tutaj 3) i pełniącą funkcję klucza szyfru:

\metroset{block=fill}
\begin{block}{Dane:  AĄBCĆDEĘFGHIJKLŁMNŃOÓPRSŚTUWYZŹŻ}
        \textbf{Szyfr:  CĆDEĘFGHIJKLŁMNŃOÓPRSŚTUWYZŹŻAĄB}
      \end{block}
\end{frame}

\begin{frame}{Przykład}
	Należy przy tym zauważyć, że ostatnim literom alfabetu w górnym ciągu odpowiadają początkowe litery w ciągu dolnym (alfabet został „zawinięty”).
	
	 Chcąc zaszyfrować wiadomość, należy każdą jej literę zastąpić odpowiednikiem z szyfru (wiadomość w przykładzie jest zapisana wersalikami, aczkolwiek szyfr jest niewrażliwy na wielkość liter):


\metroset{block=fill}
\begin{block}{Tekst jawny: MĘŻNY BĄDŹ, CHROŃ PUŁK TWÓJ}
      \textbf{Tekst szyfr: OHBÓŻ DĆFĄ, EKTRP ŚZŃM YŹSŁ}
      \end{block}
\end{frame}




\begin{frame}{Ujęcie matematyczne}
	Operację szyfrowania i deszyfrowania można wyrazić w języku arytmetyki modularnej.
	
	 W tym celu wystarczy każdej literze alfabetu jednoznacznie przypisać jej numer według schematu A-0, Ą-1, B-2, …, Ż-31.
	
	 Wygodnie jest też przyjąć, że klucz \textit{n} jest pewną liczbą z zakresu 0...31 (jest to numer zaszyfrowanej litery A).
\end{frame}

\begin{frame}{Ujęcie matematyczne}

	\textbf{Szyfrowanie i Deszyfrowanie:}
	
	x - numer litery tekstu jawnego w alfabecie
	
	n - liczba pozycji do przesuniecia litery tekstu jawnego
	
	s - zaszyfrowany znak
	
	Szyfrowanie:
	
	$s = x + (n (mod 32))$
	
	Deszyfrowanie:
	
	$x = s - (n (mod 32))$
	
	
\end{frame}

\section{Algorytm Szyfrowania RSA}

\begin{frame}[fragile]{Podstawowe Informacje}
      \textbf{RSA} – jeden z pierwszych i obecnie najpopularniejszych asymetrycznych algorytmów kryptograficznych z kluczem publicznym, zaprojektowany w 1977 przez Rona Rivesta, Adi Shamira oraz Leonarda Adlemana.
      
       Pierwszy algorytm, który może być stosowany zarówno do szyfrowania jak i do podpisów cyfrowych.
       
       Bezpieczeństwo szyfrowania opiera się na trudności faktoryzacji dużych liczb złożonych.
       
        Jego nazwa pochodzi od pierwszych liter nazwisk jego twórców
\end{frame}

\begin{frame}{Kryptosystem RSA}
  Aytmetyka modulo i właściwości liczb pierwszych są podstawą szyfru RSA - techniki szyfrowania asymetrycznego z kluczem publicznym.
  
  Proces szyfrowania wyglądają następująco: Uzytkownik X wybiera dwie dowolne, duże liczby pierwsze \textit{p} i \textit{q} (im są większe, tym lepiej, gdyż od ich wielkosci zalezy bezpieczeństwo szyfru). Aby uprościc obliczenia, w poniższym przykładzie wybrane zostały liczby 11 i 17.
  
  Następnie X oblicza liczę \textit{n} będącą iloczynem tych dwóch liczb (\textit{n} = 187).
  Trzeba jeszcze wybrać liczbę \textit{e}, która wspólnie z iloczynem n stanowić będzie klucz szyfrujacy.
  
   Liczby \textit{e} oraz (\textit{p}-1)*(\textit{q}-1) powinnym być względnie pierwsze. Załóżmy, że \textit{e} = 7.

\end{frame}

\begin{frame}{Kryptosystem RSA}
	Obie te liczby (\textit{n} i \textit{e}) podawane są do wiadomości wszystkkich, którzy chcieliby wysłac zaszyfrowaną wiadomość do danego użytkownika.
     Aby to zrobić, należy w pierwszej kolejności przekształcić wysłaną wiadomość do postali liczbowej.

	Stosowany jest tu standard ASCII. Niech przesłanym znakiem będzie litera A. W kodzie ASCII symbolowi temu odpowiada kombinacja 01000001. Po przeliczeniu na system dziesiętny przyjmuje ona wartośc 65.
	
  Tekst jawny wynosi więc \textit{J} = 65.	
	
\end{frame}

\begin{frame}{Kryptosystem RSA}
	Wartość tekstu tajnego obliczana jest według wzoru:
	
	$S=J^e(mod\textit{n})$
	
	a zatem
	
	$S = 65^7(mod187)=[65^4(mod187)*65^2(mod187)*65(mod187)]mod187$
	
	W celu usprawnienia obliczeń korzysta się tu z własności potęgowania:
	
	$a^{n+m} = a^n * a^m$
	
	oraz z własności arytmetyki modulo:
	
	$(a*b)modn = ((a mod n)*(b mod n))mod n$
	
	czyli:
	
	$[17850625(mod187)*4225(mod187)*65]mod187 =
	(166*111*65)mod187=[7215(mod187)*166(mod187)]mod187=
	109*166(mod187)=142$
	
	Wiadomość po zaszyfrowaniu ma zatem postać \textit{S} = 142.
\end{frame}

\begin{frame}{Kryptosystem RSA}
  Warto zauważyć, że dla kogoś, kto posiada tylko publiczne klucze, odwrócenie działania funkcji jest niemożliwe.
  
   Nawet jeśli zna on algorytm szyfrowania, nie jest w stanie powiedzieć, jakie były dane wejściowe, ponieważ 142 = 329(mod187) lub 471(mod187) lub 163(mod187) itd. Dopiero wartość $65^7$(mod187) jest poprawna.
   
    Aby sprawdzić całą przestrzeń kluczy i dojść aż do tej wartości w rozsądnym czasie potrzeba ogromnej mocy obliczeniowej.
\end{frame}
\begin{frame}{Kryptosystem RSA}
  Aby odwrócić działanie funkcji szyfrującej, należy uruchomic jej "zapadkę", czyli klucz deszyfrujący. Użytkownik X oblicza go ze wzoru:
  
  $ke=1mod(p-1)(q-1)$
  
  Tak więc:
  
  $7k = 10*16+1=161$
  
  stąd :
  
  $k=161/7=23$
  
  Teraz użytkownik X używa swojego tajnego klucza, aby odczytać wiadomość.Dekryptaż odbywa się według poniższego wzoru:
  
  $j=S^k(modn)$
  
\end{frame}

\begin{frame}{Kryptosystem RSA}
  Wstępne obliczenia wyglądają nastepująco:
  
  $J=142^{23}(mod187)=[142^{16}(mod187)*142^4(mod187)*142^2(mod187)*
  142^1(mod187)]mod(187)$
  
  
  $142^1(mod187)=142$
  
  $142^2(mod187)=155$
  
  $142^4mod187=(142^2mod187)^2mod187=155^2mod187=89$
  
  $142^8mod187=(142^4mod187)^2mod187=89^2mod187=67$
  
  $142^{16}mod187=(142^8mod187)^2mod187=67^2mod187=1$
\end{frame}

\begin{frame}{Kryptosystem RSA}
  Ósma potęga liczby 142 nie wystepuje we wzorze obliczającym wartość tekstu jawnego, jej wyznaczenie ułatwia jednak obliczenie wartości $142^{16}mod187$.
  
   Wstawiamy teraz obliczone reszty modulo kolejnych potęg do wzoru głównego:
  
  $142^{23}(mod187)=(1*89*155*142)mod187=[(89*155mod187)*$
  
  
  $(142mod187)]mod187=(144*142)mod187=20448mod187=65$
\end{frame}


\begin{frame}{Tabelka}
  \begin{table}
   \caption*{\textbf{\Large Długość i zastosowanie kluczy szyfrujących}}
    \begin{tabular}{@{} lr @{}}
      \toprule
      Dłygośc klucza & Zastosnowanie\\
      \midrule
      $512b < 2048b$  & brak\\
      $2048b < 4096b$& dla cywilów\\
      $4096b+$ & wojskowe\\
      \bottomrule
    \end{tabular}
  \end{table}
\end{frame}

\section{MD5}

\begin{frame}{Podstawowe informacje}
  MD5 (z ang. Message-Digest algorithm 5) – algorytm kryptograficzny, opracowany przez Rona Rivesta (współtwórcę RSA) w 1991 roku, będący popularną kryptograficzna funkcją skrótu, która z ciągu danych o dowolnej długości generuje 128-bitowy skrót.
  
  Funkcja skrótu, funkcja mieszająca lub funkcja haszująca – funkcja przyporządkowująca dowolnie dużej liczbie krótką, zawsze posiadającą stały rozmiar, niespecyficzną, quasi-losową wartość, tzw. skrót nieodwracalny.

W informatyce funkcje skrótu pozwalają na ustalenie krótkich i łatwych do weryfikacji sygnatur dla dowolnie dużych zbiorów danych.
\end{frame}

\begin{frame}{Przykład}
  Skrót obliczony dla krótkiego tekstu:
  
	\metroset{block=fill}
	\begin{block}{MD5("Ala ma kota") = 91162629d258a876ee994e9233b2ad87}
    \end{block}
    
    Nawet niewielka zmiana w tekście (w tym przypadku zamiana a na y) powoduje (z 		bardzo dużym prawdopodobieństwem) powstanie zupełnie innego skrótu MD5
	
	\metroset{block=fill}
	\begin{block}{MD5("Ala ma koty") = 6a645004f620c691731b5a292c25d37f}
    \end{block}
\end{frame}

\begin{frame}{Przykład}
  	Dość powszechnym zastosowaniem MD5 jest generowanie skrótów wszelkiego rodzaju plików udostępnianych publicznie (najczęściej w Internecie), dzięki czemu osoba, która pobrała dany plik z sieci może od razu zweryfikować (generując skrót MD5 na swojej kopii i porównując wyniki) czy jest to ten sam plik, który został zamieszczony przez jego autora lub czy nie nastąpiły przekłamania podczas samego procesu pobierania danych.
  	
  	 Publikowana w takim przypadku wartość ma postać 32-znakowej liczby w zapisie szesnastkowym.

\metroset{block=fill}
	\begin{block}{Wynik MD5 dla archiwum "linux-2.6.10.tar.bz2" o wielkości 35 MB: cffcd2919d9c8ef793ce1ac07a440eda}
    \end{block}

\end{frame}

\begin{frame}{Bibliografia}

Witryny internetowe:

\url{https://pl.wikipedia.org/wiki/Kryptologia}
\href{https://pl.wikipedia.org/wiki/Kryptologia}{Kryptologia}

\url{https://pl.wikipedia.org/wiki/Szyfr_Cezara}
\href{https://pl.wikipedia.org/wiki/Szyfr_Cezara}{Szyfr Cezara}

\url{https://pl.wikipedia.org/wiki/RSA_(kryptografia)}
\href{https://pl.wikipedia.org/wiki/RSA_(kryptografia)}{RSA}

\url{https://pl.wikipedia.org/wiki/MD5}
\href{https://pl.wikipedia.org/wiki/MD5}{MD5}

Literatura:

Podstawy kryptografii. Wydanie III - Marcin Karbowski

\end{frame}

\begin{frame}[standout]
	Dziękuję za uwagę
\end{frame}

\appendix



\end{document}
